%%%%%%%%%%%%%%%%%%%%%%%%%%%%%%%%%%%%%%%%%%%%%%%%%%%%%%%%%%%%%%%%%%%%%
\documentclass{anstrans}
%
%  various packages that you may wish to activate for usage 
\usepackage{graphicx}
\usepackage{tabls}
\usepackage{hyperref}
\hypersetup{breaklinks=true, linkcolor=black, citecolor=black,
  urlcolor=black, colorlinks=true, pdftitle={PyNE Verification and Validation}}
\usepackage{listings}
\usepackage{booktabs}
\usepackage{multirow}
\usepackage{array}
%

%General Short-Cut Commands
\newcommand{\superscript}[1]{\ensuremath{^{\textrm{#1}}}}
\newcommand{\subscript}[1]{\ensuremath{_{\textrm{#1}}}}
%\newcommand{\nuc}[2]{\superscript{#2}{#1}}
\newcommand{\nuc}[2]{{#1}-{#2}}
\newcommand{\ith}[0]{$i^{\mbox{th}}$ }
\newcommand{\jth}[0]{$j^{\mbox{th}}$ }
\newcommand{\kth}[0]{$k^{\mbox{th}}$ }
\newcommand{\us}[0]{$\mu$s }

% New definition of square root:
% it renames \sqrt as \oldsqrt
\let\oldsqrt\sqrt
% it defines the new \sqrt in terms of the old one
\def\sqrt{\mathpalette\DHLhksqrt}
\def\DHLhksqrt#1#2{%
\setbox0=\hbox{$#1\oldsqrt{#2\,}$}\dimen0=\ht0
\advance\dimen0-0.2\ht0
\setbox2=\hbox{\vrule height\ht0 depth -\dimen0}%
{\box0\lower0.4pt\box2}}

\title{Verfication and Validation of the PyNE Open-Source Nuclear Engineering Toolkit}

\author{Elliott~Biondo$^{1}$, Anthony~Scopatz$^{1}$, Matthew Gidden$^{1}$, Rachel Slaybaugh$^{2}$}

\institute{
\and $^{1}$ The University of Wisconsin-Madison, 1500 Engineering Drive, Madison, WI 53706\\
\and $^{2}$ The University of California, Berkeley, 2521 Hearst Ave, Berkeley, CA 94709 \\
}

\email{biondo@wisc.edu}

%%%%%%%%%%%%%%%%%%%%%%%%%%%%%%%%%%%%%%%%%%%%%%%%%%%%%%%%%%%%%%%%%%%%%
%
%   BEGIN DOCUMENT
%
%%%%%%%%%%%%%%%%%%%%%%%%%%%%%%%%%%%%%%%%%%%%%%%%%%%%%%%%%%%%%%%%%%%%%
\begin{document}

\section{Introduction}

PyNE \cite{scopatz_pyne} (formally Python for Nuclear Engineering) is a
transinstitutional, open source project consisting of a collection of
computational tools pertinent to nuclear engineering analysis and simulations.
The capabilities of PyNE include canonical nuclide and reaction naming
conventions, material handling, nuclear data and cross-section reading, mesh
operations, and physics-code-specific input and output parsing. 

Within the nuclear industry, many applications require software to that meets
the quality assurance (QA) criteria set forth by the U.S. Nuclear Regulatory
Commission (NRC). The NRC endorses the American Society of Mechanical Engineers
(ASME) NQA-1-2008 regulatory standard \emph{Quality Assurance Requirements for
Nuclear Facility Applications} (Parts I and II) \cite{nrc} with the NQA-1a-2008
addendum \cite{add} for the design and construction licensing phases of nuclear
power plants and fuel reprocessing facilities \cite{nrc}. The portions of these
documents pertaint to software development and documentation outline a software
engineering development model that is incongruent with the community-orienteed
software developement model adopted by PyNE, which presents novel change of
bringing open-source, community-developed, code up to regulatory standards.

<Insert Anthony's portion.>
Three Clause-BSD.


over 300,000 lines of code and documentation and over 700 unit tests.


Organizations that seek to comply with NQA-1-2008/NQA-1-2009a can use any
portion of PyNE as a component of their design and or operations software by
complying with Part II Subpart 2.7 Par. 302 Otherwise Aquired Software. This
section provides provisions the use of "... freeware ... " that "has not been
previously approved under a program consisent with [the NQA-1 standard]." In
addition, PyNE seeks to demarcate modules for which the developement process
and documentation are fully compliant with such standards. This paper seeks to
address how PyNE software developement practices align to these standards, and
the criteria designed to delinate which portions of the PyNE codebase are fully
compliant.


\section{PyNE Software Development Practices}

The PyNE software development workflow is documented, systematic, and not
unlike that of prominent and well-establised projects such as the Linux kernel.
The PyNE workflow is centered around Git distributed version control software
and additional features provided by the Github Git reposition hosting service.
All PyNE code and documenation is stored in a public reposition on Github. The
\texttt{develop} and \texttt{master} branches of this repository represent a
software baseline: a collection of software that gone through the review and
approval process and cannot be changed without going though the change process.

In order to make changes (e.g. new features, documentation improvements, and
bug fixes), a development must first make changes on his/her own version of the
repository (fork) and issue a Github \emph{pull request} (PR). A PR is a
request for changes from a developer's fork to be applied to the main
repository. The acceptance of PR is contigent on a formal review process where
another developer (who did not participate in the writting of the changes)
reviews the changes, line-by-line, verifying the changes are consistant with
the requesters description of the changes and compliant with the PyNE style
guide. In addition, the reviwer must ensure that that all unit tests pass, and
that additional features impliemented by the PR are matches with additional
unit tests. PRs are generally iterative processes were requesters update
requests based of off the critism of the reviewer. Once the reviewer, or
reviewer(s) are satisfied, the PR is accepted, and the newest version of the
code becomes the baseline.

<Matt's portion on continuous integration.>

Dev guide/Style guide
API documentation
User's guide

Github issues

VnV warnings.

The reviewing V\%V warnings 
V\&V is both a process and collection of documentation. We follow the practice daily. We won't remove import gaurds until all documentation is preceent.

\section{Addressing NQA-1-2008/NQA-1a-2009}

NQA-1-2008/NQA-1a-2009 contain two sections pertainent to nuclear engineering
software development. In Part I: "Requirements for Quality Assurance Programs
for Nuclear Facilities," Requirement 3 Section 800 addresses ``Software Design
Control" and is unrevised by the 2009 addendum. The second portion is Part II:
"Quality Assurance Requirements for Facility Applications", Subpart 2.7:
Quality Assurance Requirements for Computer Software for Nuclear Applications.
The second part is meant as a supplement to Part I. A revise edition to Subpart
2.7 appears in full in NQA-1a-2009. The software development practices of PyNE
can be mapped to the criterion set for by these documents.


\subsection{NQA-1-2008 Part I Requirement 3 Section 800: Software Design Control}

\subsubsection{801 Design Process}

The design process suggested by this section coincide two the Waterfall methods.


801.1 Requirments - The first line of every doc string provide the purpose of the 
801.2 Software Design - document the design of the software - API documentation
801.3 Implementaion - use the standards an conventions of the organization - we have style guide and dev guide
801.4 Software Design Verification - must be preformed by an individual that did not write the code  - we have pull request code reviews.
801.5 Computer program testing - we have unit tests and CI

\subsubsection{802 Software Configuration Managment}

802.1 Configuration Identification - be able to identify each revision and diff them - git version control
802.2 Change Control - document and justify all changes - as stated in the PyNE dev guide, all changes must be made via PR. We only give push access to individualy who prove adept and consent to this responsibilty.
802.3 Status Control - that status of each configuration (version) and documenting changes that are improved but not implimented - we have verison control and also github issues.


\subsection{NQA-1a-2009 Part II: Subpart 2.7}

\subsection{Section 200 General Requirements}

201 Documentation - baseline documents retained as record.
202 Review  - two are required: 1) the process leading to acceptance testing, and verifying acceptance testing works. However this can be done at the same time.
203 Software Configuration Management - documentation, programs, and support software must be controlled, as well as the process of approving and accepting changes. We version control our code and our documentation.
204 Problem reporting and corrective action - we have github "bug" issues and a user and dev mailing list.

\subsection{300 Software Acquistion}
301 Procured Software and Aquired Service - does not apply, we get everything for free
302 Otherwise Acquired Software - understand the capabilities and limitations of the software, demostrate them, instruction for use within limits, justify any exceptions to the V\&V workflow

we've got a lot of dependencies -- our unit tests are there regression tests.

\subsection{400 Software Engineering Method}
Software engineering methods should be documented.
- This portion may be redundent
401 Design requirements should be traceable
402 Software design verificaiton
403 Implimentation
404 Acceptance testing
405 Operation - up to the user!

\subsection{500 Standards, Conventions, and Other Work practices.}

style/dev guide


\subsection{Out of Scope}
Security
600 Support software - out of scope. we don't control what other softwared users used. 

\section{Nightly Testing and Continuous Integration}

The PyNE project utilizes a workflow involving both nightly testing and
Continuous Integration (CI) to guarantee that the code base is consistently and
constantly tested and validated. While each release of PyNE is V\&V'd, it is
important to maintain confidence with the current version of the updated code
base being actively developed. PyNE uses a branching development strategy, where
the \textit{staging} branch corresponds to the actively developed and updated
code base.

The practice of Continuous Integration is a well known software engineering
practice first conceived by Beck and Jefferies \cite{beck1998extreme} that
espouses updating shared code bases in a rapid manner in order streamline the
shared development process. A key compenent of modern CI approaches is the
automated building and testing of a code base upon reception of a proposed
change (i.e., a patch). For each proposed patch, the entire patched code base is
built on one or more platforms (e.g., a flavor of Linux, a Mac Operating System
(OS), or a Windows OS), and some number of unit, integration, and regression
tests are executed. The quantity of tests run with each patch submission is
dependent on the amount of time required for their execution, allowing for
relatively quick knowlege of success or failure. The proposed patch is then only
accepted if building and testing passes on all supported platforms and the
reviewer is content with both the patch's style and content. PyNE enforces a CI
approach for any patch that is proposed to be merged into the \textit{staging}
branch.

While CI is designed to streamline the integration process, nightly testing is a
complimentary tool that verifies and validates the code base once each 24-hour
period. Each nightly test is run on one or more platforms. On a clean version of
each platform, all dependencies are installed, and the entire code base is built
and installed. All unit, integration, and regression tests are then run. A
common practice among software projects is to update all supported develop
binaries after a successful nightly build and test execution

\subsection{PyNE's Implementation}

PyNE utilizes the Build and Test Lab (BaTLab) \cite{batlab_2014} as its general
building and testing service. BaTLab provides the basic tools required to
install a project and its dependencies, test a project, and return any built
products. While PyNE builds and tests both the nightly builds and CI using
Ubuntu 12(.04) and MacOS 10.8, supports all of the following platforms:

\begin{table}[ht]
  \begin{center}
    \caption{\label{tab::batlab}Build and testing platforms supported by BaTLab. 
      Those utilized by PyNE are denoted with an asterix (*).}
    \begin{tabular}{|c|c|c|}
    \hline
    Platform & x86 Arch & Version \\
    \hline
    \multirow{2}{*}{Debain} 
    & \multirow{3}{*}{32-bit} & 5.0 \\ & & 6.0 \\ & & 7.0 \\ 
    \cline{2-3}
    & \multirow{1}{*}{64-bit} & 6.0 \\
    \hline
    \multirow{1}{*}{Fedora} 
    & \multirow{3}{*}{64-bit} & 16 \\ & & 17 \\ & & 18 \\
    \hline
    \multirow{1}{*}{MacOS} 
    & \multirow{2}{*}{64-bit} & 10.7 \\ & & 10.8* \\
    \hline
    \multirow{2}{*}{RedHat Advanced Platform} 
    & \multirow{2}{*}{32-bit} & 5.x \\ & & 6.x \\ 
    \cline{2-3}
    & \multirow{2}{*}{64-bit} & 5.x \\ & & 6.x \\
    \hline
    \multirow{2}{*}{Scientific Linux} 
    & \multirow{1}{*}{32-bit} & 5.x \\ 
    \cline{2-3}
    & \multirow{1}{*}{64-bit} & 6.x \\
    \hline
    \multirow{1}{*}{Solaris} 
    & \multirow{1}{*}{64-bit} & 5.11 \\
    \hline
    \multirow{1}{*}{Ubuntu} 
    & \multirow{2}{*}{64-bit} & 10 \\ & & 12* \\
    \hline
    \multirow{2}{*}{Windows} 
    & \multirow{1}{*}{32-bit} & XP \\ 
    \cline{2-3}
    & \multirow{2}{*}{64-bit} & 7 \\ & & 8 \\
    \hline
    \end{tabular}
  \end{center}
\end{table}

PyNE uses BaTLab's services directly for its nightly testing. Scripts were
developed that nightly install PyNE's dependencies, download the current version
of PyNE's \textit{staging} branch, install PyNE, and run all tests in the PyNE
code base. Upon completion, the PyNE developer list serv is notified in case of
any failures.

The implementation of Continuous Integration is more complicated, because it
must communicate with PyNE's repository management service, GitHub, and a build
and test service. There exist a number of CI services, such as TravisCI
\cite{travis_2014}; however, many have limitations in either their number and
type of supported platforms or constrained total building and testing time. PyNE
utilizes Polyphemus \cite{polyphemus_2014}, a plugin-based CI service that can
connect with any supported front end (e.g. Github) and back end (e.g. BaTLab).

The PyNE development team runs a Polyphemus server that utilizes its BaTLab
scripts to build and test all proposed patches to PyNE's \textit{staging}
branch. A patch proposal is initiated via GitHub's \textit{pull request}
interface. The Polyphemus server is then notified, causing the launch of a
BaTLab job. Upon completion of the BaTLab job, the server is again notified and
updates Github with the result. Finally, active and prior jobs can be monitored
via a continuously running \href{http://gorgus.pyne.io/dashboard}{dashboard}.

\section{Conclusion}

The 
As software development changes, requirements need to change to reflect this.

\bibliographystyle{ans}
\bibliography{refs}
\end{document}
