%%%%%%%%%%%%%%%%%%%%%%%%%%%%%%%%%%%%%%%%%%%%%%%%%%%%%%%%%%%%%%%%%%%%%
\documentclass{anstrans}
%
%  various packages that you may wish to activate for usage 
\usepackage{graphicx}
\usepackage{tabls}
\usepackage{hyperref}
\hypersetup{breaklinks=true, linkcolor=black, citecolor=black,
  urlcolor=black, colorlinks=true, pdftitle={PyNE Verification and Validation}}
\usepackage{listings}
\usepackage{booktabs}
\usepackage{multirow}
\usepackage{array}
\usepackage{xcolor}
%

%General Short-Cut Commands
\newcommand{\superscript}[1]{\ensuremath{^{\textrm{#1}}}}
\newcommand{\subscript}[1]{\ensuremath{_{\textrm{#1}}}}
%\newcommand{\nuc}[2]{\superscript{#2}{#1}}
\newcommand{\nuc}[2]{{#1}-{#2}}
\newcommand{\ith}[0]{$i^{\mbox{th}}$ }
\newcommand{\jth}[0]{$j^{\mbox{th}}$ }
\newcommand{\kth}[0]{$k^{\mbox{th}}$ }
\newcommand{\us}[0]{$\mu$s }

% New definition of square root:
% it renames \sqrt as \oldsqrt
\let\oldsqrt\sqrt
% it defines the new \sqrt in terms of the old one
\def\sqrt{\mathpalette\DHLhksqrt}
\def\DHLhksqrt#1#2{%
\setbox0=\hbox{$#1\oldsqrt{#2\,}$}\dimen0=\ht0
\advance\dimen0-0.2\ht0
\setbox2=\hbox{\vrule height\ht0 depth -\dimen0}%
{\box0\lower0.4pt\box2}}

\title{Verification and Validation of the PyNE Open Source Nuclear Engineering Toolkit}

\author{Elliott~Biondo$^{1}$, Anthony~Scopatz$^{1}$, Matthew Gidden$^{1}$, Rachel Slaybaugh$^{2}$}

\institute{
\and $^{1}$ The University of Wisconsin-Madison, 1500 Engineering Drive, Madison, WI 53706\\
\and $^{2}$ The University of California, Berkeley, 2521 Hearst Ave, Berkeley, CA 94709 \\
}

\email{biondo@wisc.edu}

%%%%%%%%%%%%%%%%%%%%%%%%%%%%%%%%%%%%%%%%%%%%%%%%%%%%%%%%%%%%%%%%%%%%%
%
%   BEGIN DOCUMENT
%
%%%%%%%%%%%%%%%%%%%%%%%%%%%%%%%%%%%%%%%%%%%%%%%%%%%%%%%%%%%%%%%%%%%%%
\begin{document}

\section{Introduction}

PyNE \cite{scopatz_pyne} (formerly Python for Nuclear Engineering) is a
transinstitutional, open source project consisting of a collection of
computational tools pertinent to nuclear engineering analysis and simulations.
The capabilities of PyNE include canonical nuclide and reaction naming
conventions, material handling, nuclear data and cross-section reading, mesh
operations, and physics-code-specific input and output parsing. % add to this list

Within the nuclear industry, many applications require software that meets
the quality assurance (QA) criteria set forth by the U.S.\ Nuclear Regulatory
Commission (NRC). The NRC endorses the American Society of Mechanical Engineers
(ASME) NQA-1-2008 regulatory standard \emph{Quality Assurance Requirements for
Nuclear Facility Applications} (Parts I and II) \cite{nrc} with the NQA-1a-2009
addendum \cite{add} for the design and construction of nuclear power plants and
fuel reprocessing facilities \cite{nrc}. 
In order for PyNE to be broadly adopted by the nuclear community, it must meet these rigorous software-development standards.
%The portions of these
%documents pertinent to software development and documentation outline a
%software engineering development model that is structured %somewhat differently
%from the community-oriented software development model adopted by PyNE. 
In this work we will discuss how the philosophy and implementation of PyNE's development can be consistent with the NQA-1 standards.

\textcolor{red}{Insert Anthony's portion on different software development
workflows e.g. waterfall, agile. The first sentence of the next paragraph needs
up be revised for flow but the one above looks pretty solid.}

Organizations seeking to comply with NQA-1-2008/NQA-1-2009a can use any
portion of PyNE (or code from any source for that matter) as a component of
their design and/or operations software by complying with Part II Subpart 2.7
Par.\ 302, ``Otherwise Acquired Software.'' This section provides provisions for the
use of ``freeware'' that ``has not been previously approved under a
program consistent with [the NQA-1 standard].'' To facilitate its use, PyNE seeks to
designate modules for which the development process and documentation are
fully compliant with such standards. The general PyNE software
development practices accomplish the same goals as these standard. When a particular
module reaches a logical point of completion, a few additional steps are
taken to make the module fully compliant, and this module can be labeled accordingly.

This paper presents the strategy implemented by PyNE to ensure that
open-source, community-developed code is developed in a manor compliant with
regulatory standards. The requirements of ASME-1-2008/ASME-1a-2009 are
addressed explicitly. Finally, the additional documentation necessary for full
compliance is discussed. These efforts facilitate the potential deployment of
PyNE code to a broader subset of the nuclear engineering field.

%% ADD:
% emphasis that even complying in-spirit is great!
% Github keeps everything forever.

\section{PyNE Software Development Practices}
% Why is there so much space around this heading and no others? Why do these paragraphs have an extra break when those in other sections do not?

The PyNE software development workflow is meticulous, systematic, and similar
to workflows of prominent and well-established projects such as the Linux kernel.
The PyNE workflow is centered around the Git~\cite{git_2014} distributed version control software
and additional features provided by the Github repository hosting service.
The \texttt{develop} and \texttt{master} branches of this repository represent
a software baseline: a collection of software that has gone through the review and
approval process and cannot be changed without going though an agreed-upon, formal change process.
All PyNE code, tests, and documentation are stored in a public
repository on Github. The PyNE website (\texttt{http://pyne.io}) is home to the
rendered documentation, which includes a user's guide, developer's guide, style
guide, theory manual, and application programming interface (API) documentation.
Question about PyNE are posed on the PyNE user and developer mailing lists.

In order to make changes (e.g.\ new features, documentation improvements, and
bug fixes), a developer must first make changes in his/her own version of the
repository (a fork in Github terminology) and then issue a Github \emph{pull
request} (PR) accompanied by a description of the changes made. A pull request
is a request for changes from a developer's fork to be applied to the main
repository. The acceptance of a PR is contingent upon a formal review process.

In the formal review process, another developer who did not participate in the writing of the changes
reviews the changes line-by-line, verifying the changes are (1) consistent with
the requester's description and (2) compliant with the PyNE
developer's guide and style guide. In addition, the reviewer must
ensure that all unit tests pass (there are more than 700) and that
any new features implemented in the PR are supported by appropriate unit
tests. The review process tends to be iterative, where requesters update their pull
requests in response to criticism of the reviewer. Once the reviewer or
reviewers are satisfied, the PR is accepted, and the newest version of the
code becomes the baseline.

% refer to changes as a "patch" as Matt does?
%300,000 lines of code and 700 unit tests

\subsection{Nightly Testing and Continuous Integration} 

In addition to human verification by code review, the PyNE
workflow includes both nightly testing and Continuous Integration (CI) to
guarantee that the code base is consistently and constantly tested.
It is important to maintain confidence with the current version of
the updated code base being actively developed. PyNE uses a branching
development strategy, where the \textit{develop} branch corresponds to the
actively developed and updated code base.

The practice of Continuous Integration is a well-known software engineering
practice first conceived by Beck and Jefferies \cite{beck1998extreme} that
espouses updating shared code bases in a rapid manner in order streamline the
shared development process. A key component of modern CI approaches is the
automated building and testing of a code base upon reception of a proposed
change (i.e., a patch). For each proposed patch, the entire patched code base is
built on one or more platforms (e.g., a flavor of Linux, a Mac Operating System
(OS), or a Windows OS), and some number of unit, integration, and regression
tests are executed. The quantity of tests run with each patch submission is
dependent on the amount of time required for their execution, allowing for
relatively quick knowledge of success or failure. The proposed patch is then only
accepted if building and testing passes on all supported platforms and the
reviewer is content with both the patch's style and content. PyNE enforces a CI
approach for any patch that is proposed to be merged into the \textit{develop}
branch.

While CI is designed to streamline the integration process, nightly testing is a
complimentary tool that verifies and validates the code base once each 24-hour
period. Each nightly test is run on one or more platforms. On a clean version of
each platform, all dependencies are installed, and the entire code base is built
and installed. All unit, integration, and regression tests are then run. A
common practice among software projects is to update all supported develop
binaries after a successful nightly build and test execution.

\subsubsection{PyNE's Implementation}

PyNE utilizes the Build and Test Lab (BaTLab) \cite{batlab_2014} as its general
building and testing service. BaTLab provides the basic tools required to
install a project and its dependencies, test a project, and return any built
products. While PyNE builds and tests both the nightly builds and CI using
Ubuntu 12.04 and MacOS 10.8, supports all of the following platforms:

\begin{table}[ht]
  \begin{center}
    \caption{\label{tab::batlab}Build and testing platforms supported by BaTLab. 
      Those utilized by PyNE are denoted with an asterisk (*).}
    \begin{tabular}{|c|c|c|}
    \hline
    Platform & x86 Arch & Version \\
    \hline
    \multirow{2}{*}{Debian} 
    & \multirow{3}{*}{32-bit} & 5.0 \\ & & 6.0 \\ & & 7.0 \\ 
    \cline{2-3}
    & \multirow{1}{*}{64-bit} & 6.0 \\
    \hline
    \multirow{1}{*}{Fedora} 
    & \multirow{3}{*}{64-bit} & 16 \\ & & 17 \\ & & 18 \\
    \hline
    \multirow{1}{*}{MacOS} 
    & \multirow{2}{*}{64-bit} & 10.7 \\ & & 10.8* \\
    \hline
    \multirow{2}{*}{RedHat Advanced Platform} 
    & \multirow{2}{*}{32-bit} & 5.x \\ & & 6.x \\ 
    \cline{2-3}
    & \multirow{2}{*}{64-bit} & 5.x \\ & & 6.x \\
    \hline
    \multirow{2}{*}{Scientific Linux} 
    & \multirow{1}{*}{32-bit} & 5.x \\ 
    \cline{2-3}
    & \multirow{1}{*}{64-bit} & 6.x \\
    \hline
    \multirow{1}{*}{Solaris} 
    & \multirow{1}{*}{64-bit} & 5.11 \\
    \hline
    \multirow{1}{*}{Ubuntu} 
    & \multirow{2}{*}{64-bit} & 10 \\ & & 12* \\
    \hline
    \multirow{2}{*}{Windows} 
    & \multirow{1}{*}{32-bit} & XP \\ 
    \cline{2-3}
    & \multirow{2}{*}{64-bit} & 7 \\ & & 8 \\
    \hline
    \end{tabular}
  \end{center}
\end{table}

PyNE uses BaTLab's services directly for its nightly testing. Scripts were
developed that each night install PyNE's dependencies, download the current version
of PyNE's \textit{develop} branch, install PyNE, and run all tests in the PyNE
code base. Upon completion, the PyNE developer mailing list is notified in case of
any failures.

The implementation of Continuous Integration is more complicated, because it
must communicate with PyNE's repository management service, GitHub, and a build
and test service. There exist a number of CI services, such as TravisCI
\cite{travis_2014}; however, many have limitations in either their number and
type of supported platforms or constrained total building and testing time. PyNE
utilizes Polyphemus \cite{polyphemus_2014}, a plugin-based CI service that can
connect with any supported front end (e.g. Github) and back end (e.g. BaTLab).

The PyNE development team runs a Polyphemus server that utilizes its BaTLab
scripts to build and test all proposed patches to PyNE's \textit{develop}
branch. A patch proposal is initiated via GitHub's pull request
interface. The Polyphemus server is then notified, causing the launch of a
BaTLab job. Upon completion of the BaTLab job, the server is again notified and
updates Github with the result. Finally, active and prior jobs can be monitored
via a continuously running \href{http://gorgus.pyne.io/dashboard}{dashboard}.


\section{Addressing NQA-1-2008/NQA-1a-2009}

NQA-1-2008/NQA-1a-2009 contain two sections pertinent to nuclear engineering
software development. In Part I: ``Requirements for Quality Assurance Programs
for Nuclear Facilities," Requirement 3 Section 800 addresses ``Software Design
Control" and is unrevised by the 2009 addendum. The second portion is Part II:
``Quality Assurance Requirements for Facility Applications", Subpart 2.7:
``Quality Assurance Requirements for Computer Software for Nuclear
Applications." A revised edition to Subpart 2.7 appears in-full in NQA-1a-2009.
The software development practices of PyNE can be mapped to the criterion set
forth by these documents.  The section heading below refer to pertinent items
within these documents.

\subsection{NQA-1-2008 Part I Requirement 3 Section 800: Software Design Control}

\subsubsection{801 Design Process}

% This portion is written presuming Anthony as already is described the Waterfall method.
The design process suggested by Section 801 is essentially the Waterfall method, which requires documentation of all of the following, described in Pars. 801.1-801.5:

\begin{enumerate} 
\item{Requirements - the scope of the capabilities implimented by the work.}
\item{Software design - the mathematical and computational methodologies employed to meet the requirements.}
\item{Implimentaion - the standards and convections agreed upon by the organization.}
\item{Software design verification - independent confirmation that requirements are met.}
\item{Computer program testing - comparsion of results produced by the work to expected or known results.}
\end{enumerate}

In PyNE, \emph{requirements} typically appear in description of changes made in
a pull request. The \emph{software design} of features is generally documented
within the function/class documenation strings that become part of the API
documentation. Documentation of more sophicated methods is provided in the PyNE
theory manual. The \emph{implemenation} is uniform throughout PyNE as described
in the user and develop guides. Pull requests constitute \emph{software design
verification} and the extensive use of unit testing and continuous integration
statisfies \emph{computer program testing}.


\subsubsection{802 Software Configuration Management}

In addition documenting the steps of the Waterfall method, additional
documentation must address configuration identification, configuration change
control, and configuration status control. Configuration identification
requires that each version of the code can be uniquely identified and the
difference between versions can be determined. In PyNE, all code, tests, and
documenation is version controlled with Git which full supports this
capability. Configuration change control is documentation of the changes to
software baselines including the rational for the change and appropriate
verfication. In PyNE, all changes are cataloged on Github in form closed pull
request which remain on the Github website for the lifetime of the repository.
Configuration status control requires that code can be accounted for prior to
be incorporated into the baseline and changes that are ``proposed and approved,
but no implimented'' are documented. This is accomplished by the fact all
changes to the baseline come from pull requests from developer forks, which are
version controlled in the same fashion as the baseline. In addition, Github
provides an \emph{issues} feature, which documents pending changes.


\subsection{NQA-1a-2009 Part II: Subpart 2.7}

This critera in this subpart are explicitly states to be supplimental to that
of Part I. Likewise, in the case of PyNE, many of criteria in this section are
redundent with Part I.

\subsection{Section 200 General Requirements}

Sections 201-203 describe the requirements for documentation, code review, and
software configuration management which PyNE meets by complying with Part I.
Section 204 outlines requirement for reporting software bugs and the corrective
action that ensues. A Github ``issue'' is created if bugs are identified, which
automatically notifies those subscribed to the repository notications. Github
issues also provide a mechanism for assigning bug fixes to the relevant
developer(s). 

%201 Documentation - baseline documents retained as record.
%202 Review  - two are required: 1) the process leading to acceptance testing, and verifying acceptance testing works. However this can be done at the same time.
%203 Software Configuration Management - documentation, programs, and support software must be controlled, as well as the process of approving and accepting changes. We version control our code and our documentation.
%204 Problem reporting and corrective action - we have GitHub ``bug" issues and a user and dev mailing list.

\subsection{300 Software Acquisition}

Section 301 (Procured Software and Acquired Service) does not apply to PyNE,
though PyNE does rely on code from other open source projects. Use of this code
is contigent with compliance with Section 301 (Otherwise Acquired Software).
Within PyNE, this must be done on a case-by-case basis. Some portions of PyNE
have no dependies, while other portions may require code from up to 9 other
sources. 

% 302 Otherwise Acquired Software - understand the capabilities and limitations of the software, demonstrate them, instruction for use within limits, justify any exceptions to the V\&V workflow


\subsection{400 Software Engineering Method}

Sections 401 to 404 describes criterion related to design requirements, design
verification, implimentation, and acceptance testing which are redundent with
Part I in the case of PyNE. Section 405 descibes criteria for operation, which
apply to end users, and are outside the scope of the PyNE code developement.

%Software engineering methods should be documented.
%- This portion may be redundant
%401 Design requirements should be traceable
%402 Software design verification
%403 Implementation
%404 Acceptance testing
%405 Operation - up to the user!

\subsection{500 Standards, Conventions, and Other Work practices.}

The documentation of standards, conventions, and work practices required my
this seciton is provide in the PyNE developer's guide and PyNE style guide.

\subsection{600 Support Software}
This section is also outside of the scope of PyNE, as PyNE does not control the software practices of end users.


\section{Compliance documentation.}
What is the procedure for removing V\&V warnings?


\section{Conclusion}

The 
As software development changes, requirements need to change to reflect this.

\bibliographystyle{ans}
\bibliography{refs}
\end{document}
