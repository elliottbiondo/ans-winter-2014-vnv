%%%%%%%%%%%%%%%%%%%%%%%%%%%%%%%%%%%%%%%%%%%%%%%%%%%%%%%%%%%%%%%%%%%%%
\documentclass{anstrans}
%
\usepackage{graphicx}
\usepackage{tabls}
\usepackage{hyperref}
\usepackage{breakurl}
\hypersetup{breaklinks=true, linkcolor=black, citecolor=black,
  urlcolor=black, colorlinks=true, pdftitle={PyNE Verification and Validation}}
\usepackage{listings}
\usepackage{booktabs}
\usepackage{multirow}
\usepackage{array}
\usepackage{xcolor}
\usepackage{parskip}
%
\setlength{\parskip}{0pt plus 0mm}
\setlength{\parindent}{15pt}

\title{Quality Assurance within the PyNE Open Source Toolkit}

\author{Elliott~Biondo$^{1}$, Anthony~Scopatz$^{1}$, Matthew Gidden$^{1}$, Rachel Slaybaugh$^{2}$, Cameron Bates$^{2}$, Paul P.H. Wilson$^{1}$}

\institute{
\and $^{1}$ The University of Wisconsin-Madison, 1500 Engineering Drive, Madison, WI 53706\\
\and $^{2}$ The University of California, Berkeley, 2521 Hearst Ave, Berkeley, CA 94709 \\
}

\email{biondo@wisc.edu}

%%%%%%%%%%%%%%%%%%%%%%%%%%%%%%%%%%%%%%%%%%%%%%%%%%%%%%%%%%%%%%%%%%%%%
%   BEGIN DOCUMENT
%%%%%%%%%%%%%%%%%%%%%%%%%%%%%%%%%%%%%%%%%%%%%%%%%%%%%%%%%%%%%%%%%%%%%
\begin{document}

\section{Introduction}

In order for PyNE \cite{scopatz_pyne} to be broadly adopted by the nuclear
science and engineering community, it must conform to rigorous
software development standards. In this work we discuss how the
philosophy and implementation of PyNE's development practices are consistent
with the American Society of Mechanical Engineers (ASME) NQA-1 standards
\cite{nqa}\cite{add}, and what actions will need to be taken for full
compliance.

PyNE is a trans-institutional, open source project consisting of a collection
of computational tools pertinent to nuclear engineering analysis and
simulations. The capabilities of PyNE include canonical nuclide and reaction
naming conventions, material handling, nuclear data and cross-section reading,
mesh operations, and physics-code-specific capabilities. PyNE is cognizant of
exacting regulations that apply to much of the potential user base.  Many
applications require software that meets the quality assurance (QA) criteria
set forth by the U.S.\ Nuclear Regulatory Commission (NRC). The NRC endorses
the ASME NQA-1-2008 regulatory standard \emph{Quality Assurance Requirements
for Nuclear Facility Applications} (Parts I and II) \cite{nrc} with the
NQA-1a-2009 addendum \cite{add} for the design and construction of nuclear
power plants and fuel reprocessing facilities \cite{nrc}.

These documents allude to the traditional software development workflow known as the 
\emph{waterfall} model \cite{waterfall}. This strategy features several stages:
specification (or requirements), design, implementation, verification, and maintenance. 
%Each of
%these stages is conceived of as flowing into the next stage. 
Each stage must be 
completed and reviewed before continuing on to the next stage. The waterfall 
model has been shown to be very effective for large, technical projects.
However, with the advent of distributed version control as well as physically 
and organizationally disparate development teams came the rise in popularity
of \emph{agile} development \cite{larman2004agile}. The overarching principle 
of the agile strategy is to have a short iteration cycle so that the software
may evolve and respond quickly to changing needs and use cases.

In practice, the agile method is a series of parallel waterfalls for every issue
or requirement that arises. All code must be designed, written, reviewed, 
tested, and verified. This process happens for all subsets of the code base
individually. This stands opposed to the waterfall model where the review 
process happens over the entire code base as an aggregate entity.

Agile methods are useful for teams that are small, physically separate, at
different institutions, or unfunded. This is because the overhead is much 
less than required for waterfall strategies.  Agile mechanisms allow for developers
to focus on what is interesting or useful to them on an as-needed basis.
Toolkits or library projects (rather than simulators or user interfaces) are 
particularly ripe for agile development. A well designed toolkit should be 
modular in nature which allows for parallel streams of development.  
PyNE follows an agile development strategy as much out of necessity as it
is an ideological fit.  Regardless of the development strategy, nuclear code
must adhere to the highest standards of quality.

Organizations seeking to comply with NQA-1-2008/NQA-1a-2009 can use any portion
of PyNE (or code from any source for that matter) as a component of their
design and/or operations software by complying with Part II Subpart 2.7 Section
302, ``Otherwise Acquired Software.'' This section provides provisions for the
use of ``freeware'' that ``has not been previously approved under a program
consistent with [the NQA-1 standard]'' \cite{add}.  To facilitate its use, PyNE
seeks to develop code that is fully compliant with these standards, freeing users of
any verification and validation burden.
%designate modules for which the development process and documentation are
%fully compliant with such standards. 
%The general PyNE software
%development practices accomplish the same goals as these standard. When a particular
%module reaches a logical point of completion, a few additional steps are
%taken to make the module fully compliant, and this module can be labeled accordingly.

This paper presents the strategy implemented by PyNE to ensure that
open source, community-developed code is written, reviewed, tested, and
documented in a manner compliant with regulatory standards. The requirements of
ASME-1-2008/ASME-1a-2009 are addressed explicitly. The process for declaring
PyNE code fully compliant is discussed. These efforts make the PyNE code base suitable
to a broader subset of the nuclear engineering community.

\section{PyNE Software Development Practices}

The PyNE software development workflow is meticulous, systematic, and similar
to workflows of prominent and well-established projects such as the Linux kernel.
The PyNE workflow is centered around the Git~\cite{git_2014} distributed version control software
and additional features provided by the GitHub repository hosting service.
The \texttt{develop} and \texttt{master} branches of this repository represent
a software baseline: a collection of software that has gone through the review and
approval process and cannot be changed without going though an agreed-upon, formal change process.
All PyNE code, tests, and documentation are stored in a public
repository on GitHub. The PyNE website (\texttt{http://pyne.io}) is home to the
rendered documentation, which includes a user's guide, developer's guide, style
guide, theory manual, and application programming interface (API) documentation.
PyNE also has user and developer mailing lists for questions, discussion, and feedback.

In order to make changes (e.g.\ new features, documentation improvements, and
bug fixes), developers must first make changes in their own versions of the
repository (a \emph{fork} in GitHub terminology) and then issue a GitHub \emph{pull
request} accompanied by a description of the changes. A pull request
is a request for changes from a developer's fork to be applied to the main
repository. The acceptance of a pull request is contingent upon the
successful completion of a formal review process.

In the formal review process, another developer who did not participate in the writing of the changes
reviews the changes line-by-line, verifying the changes are (1) consistent with
the requester's description and (2) compliant with the PyNE
developer's guide and style guide. In addition, the reviewer must
ensure that all unit tests pass and that
any new features implemented in the pull request are supported by appropriate unit
tests. Continuous Integration (CI)---explained below---assists in this process.
The review process tends to be iterative, where requesters update their pull
requests in response to the criticism of the reviewer. Once the reviewer or
reviewers are satisfied, the pull request is accepted, and the newest version of the
code becomes the baseline. Record of the pull request is stored by GitHub for the lifetime of the repository.

\subsection{Nightly Testing and Continuous Integration} 

PyNE uses a branching
development strategy, where the \texttt{develop} branch corresponds to the
actively developed and updated code base.
With the \texttt{develop} branch changing so frequently, it is important to maintain confidence that all tests pass on this baseline.
In addition to human verification by code review, the PyNE
workflow includes both nightly testing and CI to
guarantee that the code base is consistently and constantly tested.

The practice of Continuous Integration is a well-known software engineering
practice first conceived by Beck and Jefferies \cite{beck1998extreme} that
espouses updating shared code bases in a rapid manner in order to streamline the
shared development process. A key component of modern CI approaches is the
automated building and testing of a code base upon reception of a proposed
change (i.e., a patch). For each proposed patch, the entire patched code base is
built on one or more platforms (e.g., a flavor of Linux, a Mac Operating System
(OS), or a Windows OS), and some number of unit, integration, and regression
tests are executed. The quantity of tests run with each patch submission is
dependent on the amount of time required for their execution, allowing for
relatively quick knowledge of success or failure. The proposed patch is only
accepted if building and testing passes on all supported platforms and the
reviewer is content with both the patch's style and content. PyNE enforces a CI
approach for any patch that is proposed to be merged into the \texttt{develop}
branch.

While CI is designed to streamline the integration process, nightly testing is a
complementary tool that tests the code base once each 24-hour
period. Each nightly test is run on one or more platforms. On a clean version of
each platform, all dependencies are installed, and the entire code base is built
and installed. All unit, integration, and regression tests are then run. 
A common practice among software projects is to update all supported \texttt{develop}
binaries after a successful nightly build and test execution.

\subsubsection{PyNE's Implementation}

PyNE utilizes the Build and Test Lab (BaTLab) \cite{batlab_2014} as its platform for 
building and testing software. BaTLab provides the basic tools required to
install a project and its dependencies, test a project, and return any built
products. PyNE builds and tests both the nightly builds and CI using
Ubuntu 12.04 and MacOS 10.8 and Batlab supports many other platforms which could be utilized by PyNE.

%\begin{table}[ht]
%  \begin{center}
%    \caption{\label{tab::batlab}Build and testing platforms supported by BaTLab. 
%      Those utilized by PyNE are denoted with an asterisk (*).}
%    \begin{tabular}{|c|c|c|}
%    \hline
%    Platform & x86 Arch & Version \\
%    \hline
%    \multirow{2}{*}{Debian} 
%    & \multirow{3}{*}{32-bit} & 5.0 \\ & & 6.0 \\ & & 7.0 \\ 
%    \cline{2-3}
%    & \multirow{1}{*}{64-bit} & 6.0 \\
%    \hline
%    \multirow{1}{*}{Fedora} 
%    & \multirow{3}{*}{64-bit} & 16 \\ & & 17 \\ & & 18 \\
%    \hline
%    \multirow{1}{*}{MacOS} 
%    & \multirow{2}{*}{64-bit} & 10.7 \\ & & 10.8* \\
%    \hline
%    \multirow{2}{*}{RedHat Advanced Platform} 
%    & \multirow{2}{*}{32-bit} & 5.x \\ & & 6.x \\ 
%    \cline{2-3}
%    & \multirow{2}{*}{64-bit} & 5.x \\ & & 6.x \\
%    \hline
%    \multirow{2}{*}{Scientific Linux} 
%    & \multirow{1}{*}{32-bit} & 5.x \\ 
%    \cline{2-3}
%    & \multirow{1}{*}{64-bit} & 6.x \\
%    \hline
%    \multirow{1}{*}{Solaris} 
%    & \multirow{1}{*}{64-bit} & 5.11 \\
%    \hline
%    \multirow{1}{*}{Ubuntu} 
%    & \multirow{2}{*}{64-bit} & 10 \\ & & 12* \\
%    \hline
%    \multirow{2}{*}{Windows} 
%    & \multirow{1}{*}{32-bit} & XP \\ 
%    \cline{2-3}
%    & \multirow{2}{*}{64-bit} & 7 \\ & & 8 \\
%    \hline
%    \end{tabular}
%  \end{center}
%\end{table}

PyNE uses BaTLab's services directly for its nightly testing. Scripts were
developed that---each night---install PyNE's dependencies, download the current version
of PyNE's \texttt{develop} branch, install PyNE, and run all tests in the PyNE
code base. Upon completion, the PyNE developer mailing list is notified in case of
any failures.

The implementation of CI is more complicated, because it
must communicate with PyNE's repository management service, GitHub, and a build
and test service. There exist a number of CI services, such as TravisCI
\cite{travis_2014}; however, many have limitations in either their number and
type of supported platforms or constrained total building and testing time. PyNE
utilizes Polyphemus \cite{polyphemus_2014}, a plugin-based CI service that can
connect with any supported front end (e.g. GitHub) and back end (e.g. BaTLab).

The PyNE development team runs a Polyphemus server that utilizes its BaTLab
scripts to build and test all proposed patches to PyNE's \texttt{develop}
branch. A patch proposal is initiated via GitHub's pull request
interface. The Polyphemus server is then notified, causing the launch of a
BaTLab job. Upon completion of the BaTLab job, the server is again notified and
updates GitHub with the result. Finally, active and prior jobs can be monitored
via a continuously running \href{http://gorgus.pyne.io/dashboard}{dashboard}.


\section{Addressing NQA-1-2008/NQA-1a-2009}

NQA-1-2008/NQA-1a-2009 contains two sections pertinent to nuclear engineering
software development. In Part I: ``Requirements for Quality Assurance Programs
for Nuclear Facilities," Requirement 3 Section 800 addresses ``Software Design
Control'' and is unrevised by the 2009 addendum. The second portion is Part II:
``Quality Assurance Requirements for Facility Applications'', Subpart 2.7:
``Quality Assurance Requirements for Computer Software for Nuclear
Applications.'' A revised edition to Subpart 2.7 appears in-full in NQA-1a-2009.
The software development practices of PyNE can be mapped to the criteria set
forth by these documents. The section headings below refer to pertinent items
within these documents.

\subsection{NQA-1-2008 Part I Requirement 3 Section 800: Software Design Control}

\subsubsection{801 Design Process}

The design process suggested by Section 801 is essentially the waterfall method, which requires documentation of all of the following, described in Pars. 801.1-801.5:

\begin{enumerate} 
\item{Requirements - the scope of the capabilities implemented by the work.}
\item{Software design - the mathematical and computational methodologies employed to meet the requirements.}
\item{Implementation - writing code using the standards and conventions agreed upon by the organization.}
\item{Software design verification - independent confirmation that requirements are met.}
\item{Computer program testing - comparison of results produced by the work to expected or known results.}
\end{enumerate}

In PyNE, \emph{requirements} appear in the description of changes made in
a pull request. 
The \emph{software design} of features is documented
within the function/class documentation strings that become part of the API
documentation. Documentation of more sophisticated methods can be provided in the PyNE
theory manual as needed. The \emph{implementation} is uniform throughout PyNE as 
described in the user's and developer's guides. Pull requests constitute 
\emph{software design verification} and the extensive use of unit testing, 
continuous integration, and nightly builds
satisfy \emph{computer program testing}.

\subsubsection{802 Software Configuration Management}

In addition to documenting the steps of the Waterfall method, additional
documentation must address \emph{configuration identification}, \emph{configuration change
control}, and \emph{configuration status control}. Configuration identification
requires that each version of the code can be uniquely identified and the
difference between versions can be ascertained. In PyNE, all code, tests, and
documentation are version controlled with Git which fully and automatically 
supports these
capabilities. Configuration change control is documentation of the changes to
software baselines including the rational for the change and appropriate
verification. In PyNE, all changes are cataloged on GitHub in the form closed pull
requests which remain on the GitHub website for the lifetime of the repository.
Configuration status control requires that code can be accounted for prior to
being incorporated into the baseline and changes that are ``proposed and approved,
but not implemented'' are documented \cite{add}. This is accomplished by the fact that all
changes to the baseline come from pull requests from developer forks, which are
version controlled in the same fashion as the baseline. In addition, GitHub
provides an \emph{issue} feature, which documents pending changes.


\subsection{NQA-1a-2009 Part II: Subpart 2.7}

The criteria in this subpart are explicitly stated to be supplementary to that
of Part I. Likewise, PyNE's strategy for complying with Part I Requirement 3 will also satisfy much of Part II Subpart 2.7.

\subsection{Section 200 General Requirements}

Sections 201-203 describe requirements for documentation, code review, and
software configuration management. Using PyNE's strategy, these requirements are redundant with those described in Part I.
Section 204 outlines requirements for reporting software bugs and the corrective
action that ensues. A GitHub \emph{issue} is created if bugs are identified, which
automatically notifies the development team. GitHub
issues also provide a mechanism for assigning bug fixes to the relevant
developer(s). 

%201 Documentation - baseline documents retained as record.
%202 Review  - two are required: 1) the process leading to acceptance testing, and verifying acceptance testing works. However this can be done at the same time.
%203 Software Configuration Management - documentation, programs, and support software must be controlled, as well as the process of approving and accepting changes. We version control our code and our documentation.
%204 Problem reporting and corrective action - we have GitHub ``bug" issues and a user and developer mailing list.

\subsection{300 Software Acquisition}

Section 301: ``Procured Software and Acquired Service'' does not apply to PyNE,
though PyNE does rely on code from other open source projects. Use of this code
is contingent with compliance with Section 302: ``Otherwise Acquired Software.''
Within PyNE, this must be done on a case-by-case basis. Some portions of PyNE
have no dependencies, while other portions may require code---in varying capacity---from a multitude of
sources of various project sizes and types.

% 302 Otherwise Acquired Software - understand the capabilities and limitations of the software, demonstrate them, instruction for use within limits, justify any exceptions to the V\&V workflow


\subsection{400 Software Engineering Method}

Sections 401 to 404 describe design requirements, design
verification, implementation, and acceptance testing: reiterating the
waterfall method. NQA-1a-2009 revises this section to include security
requirements. Security requirements, and also the requirements in Section 405:
``Operation,'' are the responsibility of the end-user and are outside the scope of
PyNE.

%Software engineering methods should be documented.
%- This portion may be redundant
%401 Design requirements should be traceable
%402 Software design verification
%403 Implementation
%404 Acceptance testing
%405 Operation - up to the user!

\subsection{500 Standards, Conventions, and Other Work Practices.}

The documentation of standards, conventions, and work practices required by
this section is provided in the PyNE developer's guide and PyNE style guide.

\subsection{600 Support Software}
This section covers software tools and system software, which is also the responsibility of the end-user.

\section{Full Compliance}

Compliance with NQA-1-2008/NQA-1a-2009 requires extensive documentation.
Though the software development practices of PyNE generally satisfy the
criteria outlined by NQA-1-2008/NQA-1a-2009, formal documentation of this is
lacking in some cases. This deficiency is acknowledged in the form of Python \emph{import
warnings}. Currently, when a module of Python code from PyNE is imported by
another script for use, the user is warned that the code is not fully
compliant. As modules are developed to a logical point of completion, necessary
documentation will be completed and import warnings will be removed. 
This documentation will itemize the requirements of NQA-1-2008/NQA-1a-2009,
explain how they were met, and will then require independent approval from a
member of the PyNE development team.

\section{Conclusion}

The community-based software development model adopted by PyNE does not
preclude its use in applications that require ASME NQA-1-2008/NQA-1a-2009
compliance, which provides provisions for such software. In addition, PyNE is working towards full compliance with these standards.
Through the use of version control, pull requests, unit testing, nightly builds, and
continuous integration, PyNE software development practices accomplish the
goals of meticulous record-keeping, peer-review, validation, and accountability
demanded by the industry standard. Additional documentation of these practices for
individual modules within PyNE will make them fully compliant. These efforts ensure the accuracy and
accountability of the PyNE code base.

\section{Acknowledgements}

We would like to thank Nicholas Touran for the valuable advice he provided, which
guided many of the efforts described in this paper.

\bibliographystyle{ans}
\bibliography{refs}
\end{document}
