%%%%%%%%%%%%%%%%%%%%%%%%%%%%%%%%%%%%%%%%%%%%%%%%%%%%%%%%%%%%%%%%%%%%%
\documentclass{anstrans}
%
%  various packages that you may wish to activate for usage 
\usepackage{graphicx}
\usepackage{tabls}
\usepackage{hyperref}
\hypersetup{breaklinks=true, linkcolor=black, citecolor=black,
  urlcolor=black, colorlinks=true, pdftitle={PyNE Progress Report}}
\usepackage{listings}
\usepackage{booktabs}
%

%General Short-Cut Commands
\newcommand{\superscript}[1]{\ensuremath{^{\textrm{#1}}}}
\newcommand{\subscript}[1]{\ensuremath{_{\textrm{#1}}}}
%\newcommand{\nuc}[2]{\superscript{#2}{#1}}
\newcommand{\nuc}[2]{{#1}-{#2}}
\newcommand{\ith}[0]{$i^{\mbox{th}}$ }
\newcommand{\jth}[0]{$j^{\mbox{th}}$ }
\newcommand{\kth}[0]{$k^{\mbox{th}}$ }
\newcommand{\us}[0]{$\mu$s }

% New definition of square root:
% it renames \sqrt as \oldsqrt
\let\oldsqrt\sqrt
% it defines the new \sqrt in terms of the old one
\def\sqrt{\mathpalette\DHLhksqrt}
\def\DHLhksqrt#1#2{%
\setbox0=\hbox{$#1\oldsqrt{#2\,}$}\dimen0=\ht0
\advance\dimen0-0.2\ht0
\setbox2=\hbox{\vrule height\ht0 depth -\dimen0}%
{\box0\lower0.4pt\box2}}



\title{ASME NQA-1-2008 Verfication and Validation of the PyNE Open-Source Nuclear Engineering Toolkit}

\author{Elliott~Biondo$^{1}$, Anthony~Scopatz$^{1}$, Matthew Gidden$^{1}$, Rachel Slaybaugh$^{2}$}

\institute{
\and $^{1}$ The University of Wisconsin-Madison, 1415 Engineering Drive, Madison, WI 53706\\
\and $^{2}$ The University of California, Berkeley, 2521 Hearst Ave, Berkeley, CA 94709 \\
}

\email{biondo@wisc.edu}

%%%%%%%%%%%%%%%%%%%%%%%%%%%%%%%%%%%%%%%%%%%%%%%%%%%%%%%%%%%%%%%%%%%%%
%
%   BEGIN DOCUMENT
%
%%%%%%%%%%%%%%%%%%%%%%%%%%%%%%%%%%%%%%%%%%%%%%%%%%%%%%%%%%%%%%%%%%%%%
\begin{document}

\section{Introduction}

Be sure to note that PyNE code can only be a component of a V\&V process, which
is ultimatly carried out by an analysist. In most cases, PyNE is ``Otherwise
Acquired Software", which is subject to same criterion, but not all of it is
applicable. For example, ``Operation" and ``System Software".

NQA-1-2008 \cite{nqa} is endorsed by the U.S. Nuclear Regulatory Commission (NRC) \cite{nrc}.

\section{Addressing the Criteria of NQA-1-2008}

Part I: Requirements for Quality Assurance Programs for Nuclear Facilities
Requirement 3 of Part I is Design contronl
Section 800 of Requirement 3 is Software Design Control 
Paragraph (Par.) 801 describes the waterfall method:
801.1 Requirments
801.2 Software Design
801.3 Implimentaion
801.4 Software Design Verification
801.5

Par. 802 Software Configuration Managment
802.1 Configuration Identification
802.2 Change Control
802.3 Status Control



Part II: Quality Assurance Requirements for Facility Applications
Subpart 2.7: Quality Assurance Requirements for Computer Software for Nuclear Applications.
This is ment as a suppliment to Part I Requirement 3.


\bibliographystyle{ans}
\bibliography{refs}
\end{document}
