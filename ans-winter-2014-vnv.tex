%%%%%%%%%%%%%%%%%%%%%%%%%%%%%%%%%%%%%%%%%%%%%%%%%%%%%%%%%%%%%%%%%%%%%
\documentclass{anstrans}
%
%  various packages that you may wish to activate for usage 
\usepackage{graphicx}
\usepackage{tabls}
\usepackage{hyperref}
\hypersetup{breaklinks=true, linkcolor=black, citecolor=black,
  urlcolor=black, colorlinks=true, pdftitle={PyNE Progress Report}}
\usepackage{listings}
\usepackage{booktabs}
%

%General Short-Cut Commands
\newcommand{\superscript}[1]{\ensuremath{^{\textrm{#1}}}}
\newcommand{\subscript}[1]{\ensuremath{_{\textrm{#1}}}}
%\newcommand{\nuc}[2]{\superscript{#2}{#1}}
\newcommand{\nuc}[2]{{#1}-{#2}}
\newcommand{\ith}[0]{$i^{\mbox{th}}$ }
\newcommand{\jth}[0]{$j^{\mbox{th}}$ }
\newcommand{\kth}[0]{$k^{\mbox{th}}$ }
\newcommand{\us}[0]{$\mu$s }

% New definition of square root:
% it renames \sqrt as \oldsqrt
\let\oldsqrt\sqrt
% it defines the new \sqrt in terms of the old one
\def\sqrt{\mathpalette\DHLhksqrt}
\def\DHLhksqrt#1#2{%
\setbox0=\hbox{$#1\oldsqrt{#2\,}$}\dimen0=\ht0
\advance\dimen0-0.2\ht0
\setbox2=\hbox{\vrule height\ht0 depth -\dimen0}%
{\box0\lower0.4pt\box2}}



\title{ASME NQA-1-2008 Verfication and Validation of the PyNE Open-Source Nuclear Engineering Toolkit}

\author{Elliott~Biondo$^{1}$, Anthony~Scopatz$^{1}$, Matthew Gidden$^{1}$, Rachel Slaybaugh$^{2}$}

\institute{
\and $^{1}$ The University of Wisconsin-Madison, 1415 Engineering Drive, Madison, WI 53706\\
\and $^{2}$ The University of California, Berkeley, 2521 Hearst Ave, Berkeley, CA 94709 \\
}

\email{biondo@wisc.edu}

%%%%%%%%%%%%%%%%%%%%%%%%%%%%%%%%%%%%%%%%%%%%%%%%%%%%%%%%%%%%%%%%%%%%%
%
%   BEGIN DOCUMENT
%
%%%%%%%%%%%%%%%%%%%%%%%%%%%%%%%%%%%%%%%%%%%%%%%%%%%%%%%%%%%%%%%%%%%%%
\begin{document}

\section{Introduction}

PyNE (formally Python for Nuclear Engineering)

Be sure to note that PyNE code can only be a component of a V\&V process, which
is ultimatly carried out by an analysist. In most cases, PyNE is ``Otherwise
Acquired Software", which is subject to same criterion, but not all of it is
applicable. For example, ``Operation" and ``System Software".

NQA-1-2008 \cite{nqa} is endorsed by the U.S. Nuclear Regulatory Commission (NRC) \cite{nrc}.

Waterfall vs. Agile


\section{PyNE Software Development Practice}

Theory manual
Dev guide/Style guide
API documentation
User's guide

Verison control + Github features: issues, PR, assignment (of issue and PRs), issue tags

Continuous integration



\section{Addressing the Criteria of NQA-1-2008}

PyNE is systematic, meticulous, and downright pedantic.
This PyNE workflow as described in the dev guide is very close to V\&V. Maybe not letter for letter (which would require waterfall), but certainly in spirit! 
This may not be obvious to the uninitaited, so this paper spells it out.

V\&V is both a process and collection of documentation. We follow the practice daily. We won't remove import gaurds until all documentation is preceent.

Part I: Requirements for Quality Assurance Programs for Nuclear Facilities
Requirement 3 of Part I is Design contronl
Section 800 of Requirement 3 is Software Design Control 
Paragraph (Par.) 801 describes the waterfall method:
801.1 Requirments - The first line of every doc string provide the purpose of the 
801.2 Software Design - document the design of the software - API documentation
801.3 Implementaion - use the standards an conventions of the organization - we have style guide and dev guide
801.4 Software Design Verification - must be preformed by an individual that did not write the code  - we have pull request code reviews.
801.5 Computer program testing - we have unit tests and CI

Par. 802 Software Configuration Managment
802.1 Configuration Identification - be able to identify each revision and diff them - git version control
802.2 Change Control - document and justify all changes - as stated in the PyNE dev guide, all changes must be made via PR. We only give push access to individualy who prove adept and consent to this responsibilty.
802.3 Status Control - that status of each configuration (version) and documenting changes that are improved but not implimented - we have verison control and also github issues.


Part II: Quality Assurance Requirements for Facility Applications
Subpart 2.7: Quality Assurance Requirements for Computer Software for Nuclear Applications.
This is ment as a suppliment to Part I Requirement 3. -- In our case it is redundent, but would not necessarily be the case for all workflows.

Section 200 General Requirement
201 Documentation - baseline documents retained as record.
202 Review  - two are required: 1) the process leading to acceptance testing, and verifying acceptance testing works. However this can be done at the same time.
203 Software Configuration Management - documentation, programs, and support software must be controlled, as well as the process of approving and accepting changes. We version control our code and our documentation.
204 Problem reporting and corrective action - we have github "bug" issues and a user and dev mailing list.



300 Software Acquistion
301 Procured Software and Aquired Service - does not apply, we get everything for free
302 Otherwise Acquired Software - understand the capabilities and limitations of the software, demostrate them, instruction for use within limits, justify any exceptions to the V\&V workflow

we've got a lot of dependencies -- our unit tests are there regression tests.

400 Software Engineering Method
Software engineering methods should be documented.
- This portion may be redundent
401 Design requirements should be traceable
402 Software design verificaiton
403 Implimentation
404 Acceptance testing
405 Operation - up to the user!

500 Standards, Conventions, and Other Work practices. style/dev guide

600 Support software - out of scope. we don't control what other softwared users used. 


\section{Conclusion}


\bibliographystyle{ans}
\bibliography{refs}
\end{document}
